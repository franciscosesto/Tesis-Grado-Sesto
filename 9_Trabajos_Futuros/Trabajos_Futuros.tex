\chapter*{Posibles líneas de investigación futura}
\addcontentsline{toc}{chapter}{Posibles l\'ineas de investigaci\'on futura}
 \markright{Posibles l\'ineas de investigaci\'on futura}

Los problemas que se abordan en el contexto de esta tesis plantean una serie de interrogantes y áreas de investigación que resultan novedosas y de gran relevancia en la actualidad. Al examinar detenidamente los contenidos de cada uno de los capítulos de esta tesis, se identifican dos líneas de investigación que atraviesan todo el trabajo.

En primer lugar, se plantea la necesidad de explorar el modelo de consumo intertemporal con una función de utilidad no logarítmica. Este enfoque permitirá analizar las condiciones esenciales que deben cumplirse para optimizar dicho modelo, lo que representa un aspecto fundamental para comprender el comportamiento económico en diferentes contextos.

Un segundo aspecto crucial que se desprende de esta tesis se relaciona con la incorporación de un factor de consumo inesperado de considerable magnitud en el modelo. Esta inyección de gastos inesperados arroja luz sobre cómo los individuos reaccionan ante situaciones financieras imprevistas. Se busca, en este caso, derivar las condiciones bajo las cuales los agentes económicos seguirían prefiriendo mantener su plan inicial, incluso en presencia de estos gastos inesperados.

Estas dos áreas de investigación prometen ofrecer valiosas conclusiones sobre el comportamiento humano en términos de decisiones de consumo y planificación financiera. 

% A continuaci\'on se citan, las l\'ineas de investigaci\'on futuras.

% Con respecto a lo realizado en el cap\'itulo \ref{cap_2}: ser\'ia interesante extender los resultados num\'ericos y anal\'iticos obtenidos en este cap\'itulo a una ecuaci\'on parab\'olica completa no lineal. Esto permitir\'ia estudiar procesos de transferencia m\'as genererales, la transferencia de calor en un cuerpo anis\'otropo es un ejemplo de ello.   
   
% Con respecto a lo realizado en el cap\'itulo \ref{cap_3}: se podr\'ia abordar, con las mismas herramientas utilizadas, la estimaci\'on simult\'anea de los coeficientes de difusividad y conductividad t\'ermica. \'Este es un problema inverso muy interesante, pues permitir\'ia caracterizar materiales desconocidos someti\'endolos a un proceso simple de transferencia de calor.   

% Con respecto a lo realizado en el cap\'itulo \ref{cap_4}: resultar\'ia de inter\'es ampliar la metodolog\'ia para hallar el coeficiente de transferencia de calor para diferentes geometr\'ias de la pared disipativa. Por otra parte, extender el resultado, considerando adem\'as del proceso de convecci\'on, el de radiaci\'on. Esto permitir\'ia
% hallar un coeficiente de transferencia de calor m\'as realista que involucre a todos lo procesos de transporte de energ\'ia t\'ermica.  

% Con respecto a lo realizado en el cap\'itulo \ref{cap_5}: ser\'ia de utilidad extender el resultado a fuentes definidas como funciones espacio-temporales. Por otra parte, resultar\'ia interesante, con fines comparativos, la utilizaci\'on de diferentes operadores de regularizaci\'on. Otra l\'inea de investigaci\'on importante, es el abordaje con estas herramientas, a diferentes aplicaciones concretas; como la determinaci\'on de contaminantes en capas de agua subterr\'aneas, la identificaci\'on de fisuras en diferentes cuerpos y la localizaci\'on de c\'elulas tumorales en un tejido biol\'ogico, entre otras.

% Con respecto a lo realizado en el cap\'itulo \ref{cap_6}: Es natural la extensi\'on de los resultados obtenidos a materiales compuestos, multicapa, con $n$ interfaces s\'olido-s\'olido. Por otra parte, se podr\'ia incorporar al problema de estudio, el fen\'omeno de cambio de fase, la modelizaci\'on de estos procesos, permitir\'ia estudiar diversas situaciones m\'as generales, utilizando la teor\'ia de frontera libre.

% Con respecto a lo realizado en el cap\'itulo \ref{cap_7}: resultar\'ia \'util obtener la soluci\'on de otros problemas inversos donde se determinen diferentes par\'ametros y la extensi\'on inmediata relacionada con el cap\'itulo \ref{cap_6} donde se busque localizar las $n$ interfaces s\'olido-s\'olido.