\chapter*{Nomenclatura}
\addcontentsline{toc}{chapter}{Nomenclatura}
\markright{NOMENCLATURA}

\begin{longtable}{p{5mm} c p{120mm} }
%
\multicolumn{3}{l}{\textbf{May\'usculas} }\\
\\
$A$ & --- & Connjunto admisible de toma de datos.\\
$Ag$ & --- & S\'imbolo qu\'imico de la Plata.\\
$Al$ & --- & S\'imbolo qu\'imico del Aluminio.\\
$Cu$ & --- & S\'imbolo qu\'imico del Cobre.\\
$E_e$ & --- & Elasticidad exacta.\\
$E_m$ & --- & Elasticidad m\'inima.\\
$E_M$ & --- & Elasticidad m\'axima.\\
$E_{u}^{h}$ & --- & Elasticidad de $u$ con respecto a $h$.\\
$E_{l}^{q}$ & --- & Elasticidad de $l$ con respecto a $q$.\\
$F$ & --- & Fuente puntual $\textbf{[{$^{\circ}$}C]}$.\\
$Fe$ & --- & S\'imbolo qu\'imico del Hierro.\\
$F_{flot}$ & --- & Fuerza de flotabilidad $\textbf{[N]}$.\\
$F_N$ & --- & Fuerza neta vertical sobre el cuerpo $\textbf{[N]}$.\\
$G_r$ & --- & N\'umero adimensional de Grashof.\\
$J$ & --- & Funcional de minimizaci\'on.\\
$L$ & --- & Longitud de la barra $\textbf{[m]}$.\\
$Ni$ & --- & S\'imbolo qu\'imico del Niquel.\\
$N_u$ & --- & N\'umero adimensional de Nusselt.\\
$P$ & --- & Peso del cuerpo $\textbf{[N]}$.\\
$Pb$ & --- & S\'imbolo qu\'imico del Plomo.\\
$P_r$ & --- & N\'umero adimensional de Prandtl.\\
$Q$ & --- & Flujo t\'ermico en el \'ultimo elemento de la barra $\textbf{[W/m{$^{2}$}]}$.\\
$Q_1$ & --- & Flujo t\'ermico entrante a un elemento de la barra $\textbf{[W/m{$^{2}$}]}$.\\
$Q_2$ & --- & Flujo t\'ermico que permanece en un elemento de la barra $\textbf{[W/m{$^{2}$}]}$.\\
$Q_3$ & --- & Flujo t\'ermico saliente de un elemento de la barra $\textbf{[W/m{$^{2}$}]}$.\\
$Q_f$ & --- & Flujo t\'ermico en el fluido $\textbf{[W/m{$^{2}$}]}$.\\
$R_a$ & --- & N\'umero adimensional de Rayleigh.\\
$R_\mu$ & --- & Operador de regularizaci\'on.\\
$S$ & --- & \'Area transversal de un elemento infinitesimal de la barra $\textbf{[m{$^{2}$}]}$.\\
$S_{D}^{p_i}$ & --- & Sensibilidad de $D$ con respecto al par\'ametro $p_i$.\\
$S_{u}^{\alpha^2}$ & --- & Sensibilidad de $u$ con respecto a $\alpha^2$ $\textbf{[({$^{\circ}$}C s{$^{2}$})/m]}$.\\
$S_M$ & --- & Sensibilidad m\'axima $\textbf{[({$^{\circ}$}C s{$^{2}$})/m]}$.\\
$Sn$ & --- & S\'imbolo qu\'imico del Esta\~no.\\
$T_a$ & --- & Temperatura ambiente $\textbf{[{$^{\circ}$}C]}$.\\
$V_c$ & --- & Volumen del cuerpo $\textbf{[m{$^{3}$}]}$.\\
%
\\
\multicolumn{3}{l}{\textbf{Min\'usculas} }\\
\\
$c_p$ & --- & Calor espec\'ifico $\textbf{[J/kg {$^{\circ}$}C]}$.\\
$c_{p_f}$ & --- & Calor espec\'ifico del fluido $\textbf{[J/kg {$^{\circ}$}C]}$.\\
$d$ & --- & Di\'ametro de la barra $\textbf{[m]}$.\\
$f$    & --- & Fuente $\textbf{[{$^{\circ}$}C]}$.\\
$\tilde{f}$    & --- & Fuente $\textbf{[{$^{\circ}$}C]}$.\\
$f_n$    & --- & Desarrollo en serie de la fuente $\textbf{[{$^{\circ}$}C]}$.\\
$f_\delta$    & --- & Fuente ruidosa $\textbf{[{$^{\circ}$}C]}$.\\
$f_{\delta,\mu}$    & --- & Fuente regularizada $\textbf{[{$^{\circ}$}C]}$.\\
$g$     & --- & Aceleraci\'on gravitacional $\textbf{[m/s{$^{2}$}]}$.\\
$h$     & --- & Coeficiente de pel\'icula $\textbf{[W/(m{$^{2}$}{$^{\circ}$}C)]}$.\\
$l$ & --- & Posici\'on de interfaz $\textbf{[m]}$.\\
$l_\epsilon$ & --- & Aproximaci\'on de la posici\'on de interfaz $\textbf{[m]}$.\\
$q$ & --- & Flujo t\'ermico $\textbf{[W/m{$^{2}$}]}$.\\
$q_\epsilon$ & --- & Flujo t\'ermico ruidoso $\textbf{[W/m{$^{2}$}]}$.\\
$q_m$ & --- & Flujo t\'ermico m\'inimo $\textbf{[W/m{$^{2}$}]}$.\\
$q_M$ & --- & Flujo t\'ermico m\'aximo $\textbf{[W/m{$^{2}$}]}$.\\
$q_p$ & --- & Flujo t\'ermico promedio $\textbf{[W/m{$^{2}$}]}$.\\
$\dot{q}_{cond}$ & --- & Flujo t\'ermico conductivo $\textbf{[W/m{$^{2}$}]}$.\\
$\dot{q}_{conv}$ & --- & Flujo t\'ermico convectivo $\textbf{[W/m{$^{2}$}]}$.\\
$t$     & --- & Variable temporal $\textbf{[s]}$.\\
$t_j$     & --- & Instante de tiempo particular $\textbf{[s]}$.\\
$t_0$     & --- & Tiempo de toma de mediciones $\textbf{[s]}$.\\
$t_\infty$     & --- & Tiempo relativo al estado estacionario $\textbf{[s]}$.\\
$u$    & --- & Temperatura de la barra $\textbf{[{$^{\circ}$}C]}$.\\
$u_e$    & --- & Temperatura en estado estacionario $\textbf{[{$^{\circ}$}C]}$.\\
$u_{\epsilon}$    & --- & Medici\'on de temperatura $\textbf{[{$^{\circ}$}C]}$.\\
$u_s$    & --- & Temperatura en la superficie $\textbf{[{$^{\circ}$}C]}$.\\
$u_0$    & --- & Temperatura externa a la barra $\textbf{[{$^{\circ}$}C]}$.\\
$u_{\infty}$    & --- & Temperatura lejos de la superficie $\textbf{[{$^{\circ}$}C]}$.\\
$v$    & --- & Temperatura de la barra (funci\'on auxiliar) $\textbf{[{$^{\circ}$}C]}$.\\
$w$    & --- & Temperatura de la barra (funci\'on auxiliar) $\textbf{[{$^{\circ}$}C]}$.\\
$x$     & --- & Variable espacial $\textbf{[m]}$.\\
$\bm x$     & --- & Vector de toma de datos.\\
$x_i$     & --- & Posici\'on particular sobre la barra $\textbf{[m]}$.\\
$y$    & --- & Mediciones de temperatura $\textbf{[{$^{\circ}$}C]}$.\\
$y_\delta$    & --- & Mediciones ruidosas de temperatura $\textbf{[{$^{\circ}$}C]}$.\\
%
\\
\multicolumn{3}{l}{\textbf{Letras Griegas} }\\
\\
$\alpha^2$ & --- & Coeficiente de difusividad t\'ermica $\textbf{[m{$^{2}$}/s]}$.\\
$\hat{\alpha^2}$ & --- & Coeficiente de difusividad t\'ermica estimado $\textbf{[m{$^{2}$}/s]}$.\\
$\alpha_{A}^{2}$ & --- & Coeficiente de difusividad t\'ermica del material $A$ $\textbf{[m{$^{2}$}/s]}$.\\
$\alpha_{B}^{2}$ & --- & Coeficiente de difusividad t\'ermica del material $B$ $\textbf{[m{$^{2}$}/s]}$.\\
$\alpha_{M}^{2}$ & --- & Coeficiente de difusividad t\'ermica del material $\textbf{[m{$^{2}$}/s]}$.\\
$\beta$ & --- & Velocidad del fluido $\textbf{[m/s]}$.\\
$\beta_V$ & --- & Coeficiente de expanci\'on volum\'etrica $\textbf{[1/K]}$.\\
$\delta$     & --- & Espesor de la capa l\'imite t\'ermica $\textbf{[m]}$.\\
$\delta_M$     & --- & Ruido m\'aximo.\\
$\Delta x$     & --- & Paso espacial de discretizaci\'on $\textbf{[m]}$.\\
$\Delta t$     & --- & Paso temporal de discretizaci\'on $\textbf{[s]}$.\\
$\Delta u$     & --- & Variaci\'on de temperatura  $\textbf{[{$^{\circ}$}C]}$.\\
$\Delta E$     & --- & Variaci\'on de elasticidad .\\
$\overline{\Delta u_c}$   & --- & Variaci\'on de temperatura media en el cuerpo $\textbf{[{$^{\circ}$}C]}$.\\
$\overline{\Delta u_f}$   & --- & Variaci\'on de temperatura media en el fluido $\textbf{[{$^{\circ}$}C]}$.\\
$\epsilon$ & --- & Cota para el error de medici\'on.\\
$\epsilon_i$ & --- & Ruido simulado.\\
$\varphi$    & --- & Temperatura de la barra (funci\'on auxiliar) $\textbf{[{$^{\circ}$}C]}$.\\
$\kappa$ & --- & Coeficiente de conductividad t\'ermica $\textbf{[W/m {$^{\circ}$}C]}$.\\
$\kappa_A$ & --- & Coeficiente de conductividad t\'ermica del material $A$ $\textbf{[W/m {$^{\circ}$}C]}$.\\
$\kappa_B$ & --- & Coeficiente de conductividad t\'ermica del material $B$ $\textbf{[W/m {$^{\circ}$}C]}$.\\
$\kappa_C$ & --- & Coeficiente de conductividad t\'ermica del cuerpo $\textbf{[W/m {$^{\circ}$}C]}$.\\
$\kappa_f$ & --- & Coeficiente de conductividad t\'ermica del fluido $\textbf{[W/m {$^{\circ}$}C]}$.\\
$\kappa_m$ & --- & Coeficiente de conductividad t\'ermica m\'inimo $\textbf{[W/m {$^{\circ}$}C]}$.\\
$\kappa_M$ & --- & Coeficiente de conductividad t\'ermica m\'aximo $\textbf{[W/m {$^{\circ}$}C]}$.\\
$\nu$ & --- & Ritmo de intercambio de calor por flujo lateral $\textbf{[1/s]}$.\\
$\nu_c$ & --- & Viscocidad cinem\'atica del fluido $\textbf{[m$^{2}$/s]}$.\\
$\mu_d$ & --- & Viscocidad din\'amica del fluido $\textbf{[kg/m\,s]}$.\\
$\mu^2$     & --- & Par\'ametro de regularizaci\'on.\\
$\Theta$     & --- & Conjunto admisible de soluciones.\\
$\bm \theta$     & --- & Vector de par\'ametros desconocidos.\\
$\rho$ & --- & Densidad $\textbf{[kg/m{$^{3}$}]}$.\\
$\rho_c$ & --- & Densidad del cuerpo $\textbf{[kg/m{$^{3}$}]}$.\\
$\rho_f$ & --- & Densidad del fluido $\textbf{[kg/m{$^{3}$}]}$.\\
$\rho_{\infty}$ & --- & Densidad del fluido lejos de la superficie del cuerpo $\textbf{[kg/m{$^{3}$}]}$.\\
$\sigma^2$     & --- & Varianza.\\
$\tau_1$     & --- & Par\'ametro de estabilidad.\\
$\tau_2$     & --- & Par\'ametro de estabilidad.\\
$\tau_3$     & --- & Par\'ametro de estabilidad.\\

\end{longtable}


