\chapter*{Conclusiones generales}
\addcontentsline{toc}{chapter}{Conclusiones generales}
\markright{CONCLUSIONES GENERALES}

Los resultados generales obtenidos en esta tesis son los siguientes:

\begin{itemize}
\item Se ha formulado el modelo de consumo intertemporal en tiempo continuo considerando preferencias inconsistentes de \textcite{feigenbaum2021deviation}. Se han completado los cálculos inconclusos y se han agregado comentarios pertinentes a los resultados posteriores.
\item Se ha planteado una función de descuento cuasi-hiperbólica fundamentada en evidencia empírica, la cual captura de manera más precisa el comportamiento individual en cuanto a las preferencias futuras y su valoración. Esta función introduce un parámetro $j$ que varía entre individuos y cuyos valores son estimados a partir de investigaciones y evidencia empírica. Se ha llegado a la conclusión de que este parámetro se sitúa en el intervalo abierto (0,1).
\item Se ha observado que, a diferencia de la función de descuento hiperbólica, en la función cuasi-hiperbólica el parámetro $\rho$ óptimo de la función exponencial no es igual a $\eta$ y no puede calcularse con precisión. Sin embargo, se ha encontrado que $\rho=\eta$ es una elección subóptima que garantiza siempre una ponderación futura fuerte.
\item Se ha identificado que la función de descuento cuasi-hiperbólica cumple con el sesgo del presente. Esta característica se manifiesta en los individuos como una percepción de impaciencia a corto plazo para esperar un aumento mayor en la utilidad y una reversión de preferencias a largo plazo, donde muestran una mayor paciencia para esperar un incremento en la utilidad. 
\item Se ha llegado a la conclusión de que la función de descuento cuasi-hiperbólica satisface la condición de Pareto, la cual establece que el plan inicial de consumo es superior en términos de utilidad al plan efectivamente llevado a cabo. Debido a que el plan inicial domina en el sentido de Pareto, el individuo se comprometerá con dicho plan.  
\item	Se ha constatado que la función de descuento cuasi-hiperbólica cumple con las condiciones necesarias para generar un perfil de consumo cóncavo. Esta propiedad es suficiente para que el perfil de consumo exhiba una forma de "joroba", un patrón comúnmente observado empíricamente.
\end{itemize}


Los resultados obtenidos en el presente estudio revelan que la función de descuento cuasi-hiperbólica no solo satisface las condiciones fundamentales del modelo, sino que también representa un avance en el campo de la teoría económica y del comportamiento. Este hallazgo indica que dicha función no solo es compatible con las premisas teóricas establecidas, sino que también proporciona una representación más precisa y fidedigna de cómo los individuos valoran las utilidades o recompensas futuras. La introducción de esta nueva función de descuento es crucial para mejorar el modelo propuesto, ya que sus propiedades ofrecen una descripción más profunda del proceso de toma de decisiones intertemporal. Al reflejar de manera más fiel las preferencias temporales de los agentes económicos, ofrece una base sólida para prever con mayor precisión el comportamiento futuro de los individuos y hogares. Por ende, la inclusión de esta función de descuento cuasi-hiperbólica en el marco analítico no solo amplía nuestra comprensión teórica, sino que también tiene implicaciones prácticas significativas al mejorar la capacidad predictiva del modelo.

 