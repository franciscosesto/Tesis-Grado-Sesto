\chapter{Tasas de crecimiento relativas}\label{Apendice_C}

Este apéndice se centra en el uso de las derivadas logarítmicas para analizar la tasa de crecimiento de funciones.  Se presentaran ejemplos concretos que ilustran su utilidad en la toma de decisiones informadas y en la modelización de fenómenos complejos. 


\section{Tasas de cambio y de crecimiento}
El concepto fundamental de la tasa de cambio se define como el cociente incremental, es decir, la derivada de una función $f(x)$, la cual indica cuánto afecta un cambio infinitesimal en 
$x$ a la función. Matemáticamente, se expresa de la siguiente manera:
$$f'(x)=\dfrac{df(x)}{dx}=\lim_{h \to 0} \dfrac{f(x+h)-f(x)}{h}.$$
%
La derivada proporciona una medida del crecimiento absoluto en un punto dado. Sin embargo, cuando el interés se centra en comprender el crecimiento relativo, es decir, cuánto aumentará o disminuirá la función porcentualmente desde un valor específico, se debe adoptar un enfoque diferente.
%
Para abordar esta perspectiva de crecimiento relativo, se recurrirá a la derivada del logaritmo natural de la función $f(x)$. Esta derivada permite cuantificar el crecimiento relativo y se define de la siguiente manera:

$$\ln'(f(x))=\dfrac{f'(x)}{f(x)}$$
%
Esta expresión indica el porcentaje de cambio en relación con el valor total de la función, proporcionando así una valiosa información sobre el crecimiento relativo de la función.

\section{Ejemplo de tasa de crecimiento}
Se presenta un ejemplo que contribuirá a esclarecer la comprensión de la tasa de crecimiento. Si se considera una función $f(x)=x$:
%
$$f'(x)=(x)'=1 \Rightarrow
\ln'(f(x))=\ln'(x)=\dfrac{1}{x}.$$
%
Entonces, para determinar la tasa de crecimiento en $x=2$, se evalúa la función derivada del logaritmo y se observa que  devuelve $0.5$. Esto tiene sentido, dado que la función siempre aumenta en 1 unidad, y en el punto 2, un aumento de 1 se traduciría en un 50\%.




