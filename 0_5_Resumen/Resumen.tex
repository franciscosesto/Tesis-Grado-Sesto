\chapter*{Resumen}
\addcontentsline{toc}{chapter}{Resumen}
\markright{RESUMEN}


Esta tesis se centra en la investigación y en el desarrollo de una mejora del modelo propuesto por \textcite{feigenbaum2021deviation} en el ámbito del consumo intertemporal. El objetivo principal de este estudio es incorporar una nueva función de descuento en el modelo existente y representar de manera más precisa la percepción de los individuos en el proceso de toma de decisiones relacionado con el consumo a lo largo del tiempo.
Este trabajo se enfoca en los siguientes aspectos clave:

\begin{itemize}
\item \textit{\textbf{Formulación del modelo:}} Se han completado los cálculos inconclusos y se han proporcionado comentarios relevantes sobre los resultados del modelo de consumo intertemporal en tiempo continuo que considera preferencias inconsistentes propuesto por \textcite{feigenbaum2021deviation}. 


\item \textit{\textbf{Función de descuento cuasi-hiperbólica:}} En este trabajo, se ha incorporado al modelo de \textcite{feigenbaum2021deviation}, una función de descuento cuasi-hiperbólica introducida en \parencite{myerson1995discounting}, la cual, respaldada por evidencia empírica, proporciona una descripción más precisa del comportamiento individual en comparación con la función hiperbólica empleada en la versión original del modelo de \textcite{feigenbaum2021deviation}. Esta función, introduce un nuevo parámetro que permitirá ajustar el descuento o penalización del valor de la recompensa para retrasos largos.

\item  \textit{\textbf{Sesgo del presente:}} Se ha identificado que la función de descuento cuasi-hiperbólica satisface el sesgo del presente, lo que significa que los individuos muestran impaciencia a corto plazo para obtener una mayor utilidad, pero una mayor paciencia a largo plazo.

\item  \textit{\textbf{Condición de Pareto y perfil de consumo:}} La función de descuento cuasi-hiperbólica satisface la condición de Pareto, lo que implica que el plan inicial de consumo es preferible al plan realizado. Además, esta función de descuento es capaz de generar un perfil de consumo cóncavo, lo que se refleja en un patrón de consumo en forma de “joroba” (forma concava).

%
\end{itemize}


\hspace{1cm}














