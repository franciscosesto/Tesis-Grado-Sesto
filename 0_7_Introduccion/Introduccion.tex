\chapter*{Introducci\'on}
\addcontentsline{toc}{chapter}{Introducci\'on}
\markright{INTRODUCCI\'ON}

La teoría del consumo y el ahorro constituye una rama fundamental de la economía que en la actualidad es utilizada para explicar el comportamiento agregado de las finanzas. Inicialmente, y en su sentido más amplio, ha sido desarrollada por economistas que centran su estudio en las relaciones de variables (como ingreso, ahorro, tasa de interés y consumo), quienes tuvieron la necesidad de implementar una teoría que les permita dar respuestas sobre cómo interactúa el consumo. La investigación sobre la decisión de consumo es vital ya que las familias e individuos ven afectado su bienestar y felicidad por ello. La decisión de consumo en el momento inicial implica necesariamente la renuncia a un consumo futuro, y del mismo modo la renuncia a un consumo en el momento incial, o análogamente, ahorro implica postergarlo para un momento posterior. Como se ilustra claramente en \parencite{larrain2002macroeconomia} a nivel agregado la tasa de crecimiento económico, el equilibrio comercial, el nivel de ingresos y empleo se ve influenciado por las elecciones conjuntas de consumo y ahorro realizadas por los hogares.

Entre los modelos de consumo más significativos que aparecen en la bibliografía, se pueden mencionar: el modelo de ingreso absoluto \parencite{keynes1936general}, el modelo de ingreso permanente \parencite{friedman1957theory} y el modelo de ciclo de vida \parencite{modigliani1954utility}. Éste ofrece un enfoque temporal más completo que permite una evaluación más precisa de las políticas públicas. Al analizar cómo las familias toman decisiones de consumo y ahorro a lo largo del tiempo, los responsables de la formulación de políticas, pueden identificar áreas de vulnerabilidad económica y desarrollar estrategias para fomentar el ahorro responsable y el bienestar a largo plazo de los hogares. Tal perspectiva de largo plazo resulta primordial en un contexto económico caracterizado por la incertidumbre y la variabilidad de los ingresos y gastos a lo largo del ciclo de vida de los individuos. 

Dentro de los modelos que derivan del ciclo de vida es de suma relevancia investigar la interacción con preferencias inconsistentes y un enfoque de tiempo continuo como lo hacen \parencite{feigenbaum2021deviation} debido a que estas extensiones permiten capturar de manera más realista el comportamiento de los individuos y su impacto en la economía a lo largo del tiempo. 

En primer lugar, la consideración de preferencias inconsistentes en el modelo de ciclo de vida toma en consideración una limitación fundamental de los enfoques tradicionales que asumen preferencias consistentes. Las preferencias inconsistentes reflejan la posibilidad de que las personas tomen decisiones diferentes dependiendo de en que momento de su vida se encuentran. Esta variabilidad en las preferencias es particularmente relevante en contextos económicos inciertos lo que nos puede llevar a un mayor entendimiento de la toma de decisiones reales.

En segundo lugar, el enfoque de tiempo continuo permite reflejar escenarios más realistas donde el individuo está constantemente realizando elecciones y ponderando sus preferencias. Supera las restricciones de los modelos discretos, que consideran intervalos fijos de tiempo para la toma de decisiones. 

La predicción del comportamiento económico desempeña un papel fundamental tanto en el diseño de políticas públicas como en la comprensión de las decisiones individuales. Es especialmente relevante investigar la posibilidad de mejorar la función en el modelo que representa las percepciones de las personas, de modo que estas reflejen con mayor precisión las valoraciones y percepciones que se producen en la vida real. 
%Es esencial enfatizar que estas adiciones podrían enriquecer nuestra comprensión de la complejidad de las decisiones económicas y brindar nuevas perspectivas sobre cómo los individuos interactúan con su entorno financiero. Al expandir el enfoque para considerar una gama más amplia de comportamientos, podríamos acercarnos más a representar fielmente la diversidad de decisiones que influyen en el comportamiento agregado de la economía.

Esta tesis se enfoca en investigar la integración de una nueva función de descuento en el modelo propuesto por \parencite{feigenbaum2021deviation}, con el objetivo de representar de manera más precisa la percepción de los individuos. Esto permitirá una representación más fiel del proceso de toma de decisiones dentro del modelo.   
El objetivo principal de este estudio radica en la mejora continua del modelo propuesto. Esta mejora se lleva a cabo con la finalidad específica de emplear las salidas del modelo para predecir el comportamiento de consumo a lo largo de la vida del individuo de manera más precisa y detallada. Al perfeccionar el modelo, se busca no solo incrementar su exactitud en las predicciones, sino también capturar matices y complejidades inherentes al proceso de toma de decisiones, relacionado con el consumo a lo largo del tiempo. 

El modelo empleado en este trabajo se basa en la propuesta de \parencite{feigenbaum2021deviation} en el ámbito del consumo intertemporal. Se trata de un modelo en tiempo continuo que aborda preferencias inconsistentes. Uno de los aspectos distintivos de este modelo es su capacidad para caracterizar la función de descuento en términos de un factor de ponderación futuro. 

Esta tesis está organizada de la siguiente manera. En el capítulo \ref{cap_1}, se presenta de forma concisa una serie de conceptos fundamentales indispensables para la comprensión integral de este trabajo.

En el capítulo \ref{cap_2}, se lleva a cabo un estudio del entorno del modelo descrito por \parencite{feigenbaum2021deviation}. Con el propósito de caracterizar este entorno, se introduce la función de descuento, la cual refleja la percepción del individuo en relación a una utilidad o recompensa futura, caracterizándola en términos de un factor de ponderación futuro. Además, se plantean las condiciones necesarias para el sesgo del presente y la reversión de preferencias. Posteriormente, se aborda el problema del hogar, que se centra en la maximización de su utilidad bajo restricciones presupuestarias.

En el capítulo \ref{cap_3}, se prosigue con el estudio del modelo propuesto por \parencite{feigenbaum2021deviation}, centrándose en las condiciones que garantizan que el individuo se comprometa con el plan de consumo inicial. Se lleva a cabo una comparación entre la utilidad generada por el plan de consumo inicial y el plan de consumo efectivamente realizado y se establecen las condiciones necesarias.

En el capítulo \ref{cap_4}, se avanza en el análisis del modelo propuesto por \parencite{feigenbaum2021deviation}, poniendo especial énfasis en las condiciones que aseguran que el plan de consumo exhiba concavidad. Esta propiedad es esencial para que el perfil de consumo tenga una forma de "joroba" que concuerde con las observaciones empíricas.

En el capítulo \ref{cap_5}, se lleva a cabo un análisis exhaustivo acerca de la conceptualización del valor de la recompensa, examinando las perspectivas propuestas por diversos autores en relación con su significado y las metodologías empleadas para su medición. Posteriormente, se procede a una detenida comparación de distintas funciones de descuento, basándose en la evidencia empírica recopilada, con el objetivo de identificar aquella que mejor se adecua a dicha evidencia. Finalmente, se evalúa la función seleccionada en relación con el cumplimiento de cada una de las condiciones establecidas por el modelo, con el fin de determinar su idoneidad y pertinencia en el contexto de la investigación.
 








